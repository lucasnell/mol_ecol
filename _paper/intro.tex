

\section{Introduction}


Experimental evolution studies have been instrumental in informing our understanding
of the processes shaping evolution \citep{Elena:2003fr}. 
Most of such studies have been carried out in microbes
\citep[reviewed in][]{Dettman:2012gt,Jerison:2015kw}, and have provided insights on
such diverse and fundamental themes as historical contingency, evolutionary innovation,
parallel evolution, and adaptation
\citep{Blount:2008fo,Barrick:2009in,TollRiera:2016fd,Voordeckers:2015cl,Gerstein:2012ki}.


Similar experiments have been carried out in invertebrates
\citep{Gompert:2016eg,Chandler:2014bn,Burke:2010eq,Kang:2016ea,Rouchet:2014jl}, 
although such studies are comparatively rare.
Experimental evolution studies in insects typically utilize limited numbers of 
clonal or inbred lines \citep[e.g.,][]{Rouchet:2014jl,Kang:2016ea} and 
characterize experimental populations by either 
(a) measuring the distribution of specific phenotypes \citep[e.g.,][]{Rouchet:2014jl}
or
(b) sequencing pooled DNA \citep[``Pool-seq''; e.g.,][]{Burke:2010eq}.
The former requires that the researcher manipulate the environment such that a 
specific phenotype is predicted to change.
Moreover, if starting experimental populations have a continuous distribution of
phenotypes or if any significant degree of phenotypic plasticity exists, this method
is not likely to provide accurate estimates of how distributions of individuals change
through time.

Pool-seq, however, is an accurate, cost-effective method to measure allele frequencies
in populations \citep{Gautier:2013dv,Futschik:2010be}
and to identify loci associated with traits \citep{Rubin:2010es,Bastide:2013jx}.
However, Pool-seq's advantageous accuracy-to-cost ratio is only present when there
are many pooled individuals ($>40$) and when depth of coverage is high ($> \times 50$).
Because sequencing error is difficult to distinguish from rare alleles, 
Pool-seq is not ideal when trying to detect these low-frequency alleles
\citep{Schlotterer:2014dk}.
Additionally, Pool-seq of whole-genome sequencing provides much unnecessary 
information if an association study is not the ultimate goal.

One way to reduce genome complexity is to use restriction site-associated DNA 
sequencing (``RADseq'').
RADseq approaches use restriction enzymes to break apart the genome at specific 
locations determined by the enzyme's binding site sequence.
Although some use RADseq to refer to one specific methodology, here I use the more
inclusive definition of RADseq by
\citet{Andrews:2016bc}, who define RADseq as all methods 
``... that rely on restriction enzymes to determine the set of loci to be sequenced''
\citep[p 81]{Andrews:2016bc}.
Many iterations of RADseq exist and each has particular situations where they are most 
appropriate, such as 2bRAD or ezRAD when many cut sites are required or double-digest RAD
(ddRAD) when sampling complex genomes or needing extreme flexibility 
\citep{Andrews:2016bc}.
Genotyping-by-sequencing (``GBS'') does not usually provide as many cut sites as some 
other methods, but it is a particularly low-effort, low-cost RADseq approach
that requires no specialized equipment for sample preparation \citep{Elshire:2011gn}.
I have decided on GBS going forward, and below I will outline why its aforementioned
attributes are particularly suitable for my experiment.

The purpose of this paper is to assess pooled GBS as a method to estimate the abundance 
of clonal lines of aphids after experimental evolution.
In the proposed experiment, I seek to measure the repeatability of evolution by 
starting replicate cages (2m $\times$ 1m $\times$ 1m) with 50 each of the same 10 aphid
clones, allowing aphids to compete in cages for ~6 months, then determining the relative
abundances of the starting clonal lines at the end of the experiment.
Aspects of this specific experiment that make it suitable for pooled GBS are as follows:

\textbf{Pooling:}
Each experimental population will contain $\gg$1,000 individuals when allele frequencies 
are sought at the end of the experiment. Individual-based estimates (e.g., using 
microsatellites) would require ~50--100  individuals to be sampled from the population 
(Fig \ref{fig:sampling}), which would take huge amounts of preparation time. Pool-seq 
would allow me to only prepare a single or a few samples per  experimental 
replicate, and, given adequate coverage, should provide accurate allele frequencies 
\citep{Schlotterer:2014dk}.

\textbf{GBS:}
My lab group has limited access to wet lab supplies and cannot conduct lengthy sample
preparation procedures.
Because I also should not need many cut sites to distinguish between clonal lines, GBS
should serve my purposes adequately.


\begin{figure}[!ht]
    \centering
    \includegraphics[width=6in]{sampling_p.pdf}
    \caption{Simulated estimates of mean genotype abundance in a population for a given
        sample size. Samples were randomly drawn from a population of 1,000 with 
        $\mu = 0.5$. $N$ is the number of samples drawn from the population, and 
        distributions are for 1,000 simulations.
        See \href{https://github.com/lucasnell/mol_ecol/blob/master/samp_plot.R}{here} 
        for this figure's code.}
    \label{fig:sampling}
\end{figure}



The accuracy of pooled GBS to estimate allele frequencies from populations of known
genotypes has never been examined, to this author's knowledge.
I will measure accuracy using simulations of pooled GBS from variants derived from the 
aphid reference genome.








