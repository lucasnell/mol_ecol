
\section{Discussion}

Simulated pooled GBS data resulted in abundance estimates nearly identical to correct
values.
However, multiple sources of error were not simulated here, and would need to be if
a proper test of this method is to be made.
First, I simply filtered the pooled DNA \texttt{fastq} file to get reads for the 
individual sample reads. This means that the same sequencing error that occurred in 
the pooled reads occurred in the samples' reads, which is unrealistic.
Second, I created variants after digesting the genome, so mutations occurring in 
restriction enzyme binding sites, a common problem for RADseq data \citep{Andrews:2016bc},
was not introduced into my simulations.
Third, PCR bias was partially replicated when I filtered by fragment size, but
I did not simulate an actual PCR process. Other biases may be introduced if a full PCR
process is implemented.
Fourth, I only introduced SNPs into genomic variants, but more complex genomic 
changes (e.g., indels, inversions) may necessitate more complex analyses.
Fifth, no individuals were heterozygous in my simulations.

If pooled GBS remains reasonably accurate with these additional sources of error, there
are other tests that I need to run to optimize how I use this method.
All samples were evenly distributed in the pooled DNA, but this will not likely be
the case in reality. Assessing how rare clonal lines are detected using this approach 
will be a key goal going forward.
Read depth will be particularly important for determining rare allele frequencies, but
it might also have overall effects on allele frequency estimates.
This should also be assessed.
