
\documentclass[12pt]{article}


\usepackage[letterpaper, margin=1in]{geometry}
% \usepackage[utf8]{inputenc}
\usepackage{amsmath}


% ---------
% Paragraph formatting
% ---------
\setlength{\parindent}{0in}
\setlength{\parskip}{2ex}
\linespread{1}


% ---------
% Figures
% ---------
\usepackage{graphicx,setspace}
% Figures are another directory:
\graphicspath{{~/Google_Drive/Wisconsin/_Mol_Ecol/_paper/}}
\usepackage[
    justification=raggedright,
    margin=1in,
    % stretch=0.75,
    skip=6pt,
    labelfont=bf
]{caption}



% ---------
% Bibliography and in-text citations
% ---------
\usepackage[
    backend=biber,
    maxcitenames=2,
    style=authoryear,
    % givenits=true,
    maxbibnames=99,
    maxcitenames=2
]{biblatex}
\addbibresource{~/Dropbox/refs.bib}







\title{\vspace*{-3ex}Simulating GBS data for pooled samples}
\author{Lucas Nell}
\date{\vspace*{-3ex}\emph{Spring 2017}}

\begin{document}

\maketitle



\raggedright

Nei and Li's (1979) measure of nucleotide diversity, $\theta_{\pi}$, is calculated using
the following equation:

\begin{equation} \label{eq_neili}
    \theta_\pi = \sum_{ij} x_i x_j \pi_{ij}
\end{equation}

Here, $x_i$ and $x_j$ represent the frequencies of the $i$th and $j$th unique 
sequences respectively and 
$\pi_{ij}$ represents the proportion of divergent sequence between the $i$th and 
$j$th unique sequences.

If I assume that all lines will be unique sequences—a safe assumption if whole 
genomes are considered—then the above equation can be expressed as follows:

\begin{equation} \label{eq_unqseq}
    \theta_\pi = \frac{1}{n^2} \sum_{ij} \pi_{ij}
\end{equation}

Then, since the number of total pairwise combinations between $n$ sequences is 
$\binom{n}{2}$, we can calculate $\bar{\pi}$, the mean proportional
sequence divergence between any two sequences, as such:

% \begin{equation} \label{eq_barpi}
%     \begin{split}
%         \bar{\pi} = \frac{ \sum_{ij} \pi_{ij} }{ \binom{n}{2} } \\
%         \sum_{ij} \pi_{ij} = \binom{n}{2} \bar{\pi}
%     \end{split}
% \end{equation}


\begin{equation} \label{eq_barpi}
    \bar{\pi} = \frac{ \sum_{ij} \pi_{ij} }{ \binom{n}{2} }
\end{equation}


Some simple arithmetic gives us...

\begin{equation} \label{eq_sumij}
    \sum_{ij} \pi_{ij} = \binom{n}{2} \bar{\pi}
\end{equation}

Now I insert this into equation \ref{eq_unqseq}:

\begin{equation} \label{eq_insbpi}
    \theta_\pi = \frac{1}{n^2} \binom{n}{2} \bar{\pi}
\end{equation}

Solving for $\bar{\pi}$ yields the following:

\begin{equation} \label{eq_solvebpi}
    \bar{\pi} = \frac{\theta_\pi n^2}{\binom{n}{2}}
\end{equation}


\end{document}