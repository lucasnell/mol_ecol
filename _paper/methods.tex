
\section{Methods}


All of my code was created in R version 3.3.3 \citep{RCoreTeam:2014wf}, and C++ code was 
implemented using the \emph{`Rcpp'} package \citep{Eddelbuettel:2013if}.
Code for the simulations can be found \href{https://github.com/lucasnell/mol_ecol}{here}.

\subsection{Initial steps}

I first input the aphid reference genome fasta file into R, where I first digested it 
with the restriction enzyme of my choice (\emph{ApeKI} for now).
Longer digested fragments are not as likely to be sequenced 
\citep{Andrews:2016bc,Elshire:2011gn}, so I filtered for size based on the distribution 
of fragment sizes seen by \citet{Elshire:2011gn}.


\subsection{Creating variants}

I next created 10 versions of the aphid genome, each containing SNPs, to simulate
10 different clonal lines.
SNP abundance and characteristics where present were based on two estimates of 
population-genomic diversity from \citet{Bickel:2013hy}: $\theta_w = 0.0050$ and 
$\theta_{\pi} = 0.0045$.
From these estimates, I estimated the proportion of segregating sites and the nucleotide 
diversity at segregating sites, respectively. 
I will calculate the proportion of segregating sites first.

$\theta_w$ is calculated as follows \citep{Watterson:1975bh}:

\begin{equation} \label{eq:watterson} 
    \theta_w = \frac{ K }{ a_n }
\end{equation}

where $K$ is the proportion of segregating sites. Variable $a_n$ is below:

\begin{equation} \label{eq:a_n}
    a_n = \sum_{i=1}^{n-1} \frac{1}{i}
\end{equation}

where $n$ is the number of individuals sampled.
Thus the the proportion of segregating sites is simply $K = \theta_w a_n$.
For 10 samples, $K = 0.01414$.

Next, I will estimate the nucleotide diversity at segregating sites.
$\theta_{\pi}$ is calculated using the following equation \citep{Nei:1979hm}:

\begin{equation} \label{eq:nei}
    \theta_\pi = \sum_{ij} x_i x_j \pi_{ij}
\end{equation}

Here, $x_i$ and $x_j$ represent the frequencies of the $i$\textsuperscript{th} and 
$j$\textsuperscript{th} unique sequences respectively and $\pi_{ij}$ represents the
proportion of divergent sequence between the $i$\textsuperscript{th} and 
$j$\textsuperscript{th} unique sequences.
If I assume that all lines will be unique sequences—a safe assumption if whole
genomes are considered—then the above equation can be expressed as follows:

\begin{equation} \label{eq:unqseq}
    \theta_\pi = \frac{1}{n^2} \sum_{ij} \pi_{ij}
\end{equation}




Then, since the number of total pairwise combinations between $n$ sequences is
${n \choose 2}$, we can calculate $\bar{\pi}$, the mean proportional
sequence divergence between any two sequences, as such:

\begin{equation} \label{eq:barpi}
    \bar{\pi} = \frac{ \sum_{ij} \pi_{ij} }{ {n \choose 2} }
\end{equation}

Some simple arithmetic gives us...

\begin{equation} \label{eq:sumij}
    \sum_{ij} \pi_{ij} = {n \choose 2} \bar{\pi}
\end{equation}

Now I insert this into equation \ref{eq:unqseq}:

\begin{equation} \label{eq:insbpi}
    \theta_\pi = \frac{1}{n^2} {n \choose 2} \bar{\pi}
\end{equation}

Solving for $\bar{\pi}$ yields the following:

\begin{equation} \label{eq:solvebarpi}
    \bar{\pi} = \frac{\theta_\pi n^2}{{n \choose 2}}
\end{equation}


Since I have already calculated the proportion of segregated sites, I want the mean
divergence at segregated sites only.
To only consider segregated sites, I divide the whole expression by the proportion 
of segregated sites.
This leaves me with the proportional nucleotide divergence between two sequences
at segregating sites, $\bar{\pi}_{s}$:

\begin{equation} \label{eq:barpi_s}
    \bar{\pi}_{s} = \frac{\theta_\pi n^2}{K {n \choose 2}}
\end{equation}

For the estimates above, $\bar{\pi}_s = 0.7072$. At each segregating (i.e., SNP) site,
I inserted random nucleotide frequencies that matched closest to this pairwise
difference.

\subsection{Preparing sequences}

The last step in R was to prepare sequences for simulation.
This involved making sure all sequences were at least as long as the anticipated read 
length (100bp), adding barcodes, and removing sections of the fragments that won't be 
sequenced (inner portions far from cut sites) \citep{Elshire:2011gn,Davey:2011ip}.

\subsection{Simulating Illumina reads}

I used the ART sequencing simulator \citep{Huang:2012kq} to simulate single-end, 100bp
Illumina reads.
Because I only wanted fragments to be sequenced from the ends (i.e., no further 
digestion), I simulated paired-end reads with the mean size of DNA fragments set to
200bp (100bp from each end) with a standard deviation of zero.
I also forced R2 reads to have the same error and mapping quality profile as R1.

I simulated $100\times$ coverage for all samples across the entire genome.


\subsection{Downstream analyses}

I next aligned the pooled sample and all separated individual samples to the aphid genome
using BWA-MEM \citep{Li:2013wn}. 
Individual samples were created by filtering the \texttt{fastq} file output from ART
by sample name. Thus all the reads found in a given sample \texttt{fastq} file were also
present in the pooled \texttt{fastq} file.
Those resulting \texttt{SAM} files were summarized using the mpileup function in 
samtools, resulting in a single \texttt{mpileup} file.
That file was input to Popoolation2 \citep{Kofler:2011ds}, which creates a much more 
concise summary of the output, a \texttt{sync} file.


Figure \ref{fig:methods_overview} contains an overview of this simulation process.


\subsection{Abundance calculation}

From the sync file, I estimated how many alleles a sample could have contributed to
the pooled sample at a given location (from 0 to 2) based on what nucleotides
were present in the pooled and sample's alignment at that location.
I then summed these counts for each sample to create their total allele counts.
Abundances were each sample's allele counts divided by the total allele counts for all
samples.



\begin{figure}[!ht]
    \centering
    \includegraphics[width=6in]{sim_methods.pdf}
    \caption{Overview of methods used to simulate GBS data and summary. The gray box
        indicates steps run entirely in R, each represented by a red arrow.
        Black dashed arrows represent file inputs to a program, while solid black arrows
        show output from a program.
        Rounded rectangles are file formats, while rectangles are programs.
        ART = ART sequencing simulator, BWA = Burrows-Wheeler Aligner
        }
    \label{fig:methods_overview}
\end{figure}

\clearpage